Dijkstra's algorithm is often presented using numeric path weights.
However, it has been known for decades that Dijkstra's algorithm can be generalised to find the best paths over graphs having arc weights in a large class of semirings.
Traditional correctness proofs rely on the distributivity property of semirings and show that Dijkstra's algorithm computes the best path weights over all possible paths from a single source to all destinations.
Such a solution can also be seen as a fixed point of a certain matrix equation.
Here we present a proof in Agda that Dijkstra's Algorithm can solve such matrix equations, even when distributivity does not hold.
Fixed points to matrix equations for non-distributive algebras can be interpreted as representing `locally optimal' solutions.
We make extensive use of dependent types, indexed families of types, and some of Agda's more cutting edge features---such as induction-recursion---to structure our algorithm and proof of correctness.

\keywords{Dijkstra's algorithm, shortest paths, Semirings, Sobrinho Algebras, internet routing, interactive theorem proving, Agda}
