Dijkstra's Algorithm is typically presented as working over graphs with numeric arc weights, but a more general form of the algorithm exists that works over graphs where arc weights are drawn from a large class of semirings.
Traditional correctness proofs rely on the distributivity property of semirings and show that Dijkstra's algorithm computes the best path weights over all possible paths from a single source to all destinations.
Algebraically, such a solution can also be seen as a fixed point of a certain matrix equation.
In this paper, we present a formal proof that Dijkstra's Algorithm can solve these matrix equations, even when distributivity does not hold.
Fixed points to matrix equations for non-distributive algebras can be interpreted as representing `locally optimal' solutions.
Our proof is carried out in Agda, and we make use of dependent types and some of Agda's more cutting edge features---such as induction-recursion---to structure our algorithm and correctness proof.

\keywords{Dijkstra's Algorithm, semirings, interactive theorem proving}
