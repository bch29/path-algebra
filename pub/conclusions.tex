In this paper we have presented a purely functional implementation of a generalised shortest path algorithm, and proved it correct using an algebraic method.
We have made extensive use of dependent types, and some of Agda's more advanced features, such as induction-recursion, to help structure the implementation and proof.
All implementation files, and supporting documentation, are available anonymously from a public \texttt{git} repository\footnote{\url{https://bitbucket.org/curiousleo/path-algebra}}.
Our implementation consists of approximately 2,400 lines of Agda, and was developed with Agda~2.4.2.1 and~2.4.2.2 and Standard Library version~0.9.

\paragraph{Related work} Despite the algorithm's notability, Chen seems to have been the first to verify the correctness of Dijkstra's Algorithm in any proof assistant, producing a Mizar implementation in 2003~\cite{chen:dijkstra:2003}.
Later, Moore and Zhang verified Dijkstra's Algorithm in ACL2~\cite{moore:proof-pearl:2005}.
Gordon, Hurd, and Slind verified Dijkstra's reachability algorithm in HOL4 as part of a wider formalisation of Accellera~\cite{gordon:executing:2003}.
Fleury verified Floyd's all-pairs shortest path algorithm in Coq~\cite{fleury:implantation:1990}, and in unpublished work, Paulin and Fill\^iatre later verified Floyd's algorithm again in imperative form with the aid of an additional tool, also in Coq.
Nordhoff and Lammich verified Dijkstra's algorithm in Isabelle/HOL as a showcase of the Isabelle refinement and collections frameworks~\cite{nordhoff-dijkstra-2012}.
Lammich also later implemented and verified an imperative version of Dijkstra's Algorithm in Isabelle/HOL~\cite{lammich:refinement:2015}.
These prior formalisations take a classical approach to shortest path algorithms and their proofs of correctness, following textbook treatments, in contrast to our work presented in this paper.
We believe that we are the first to present a verified, axiom-free implementation and proof of correctness of a shortest path algorithm employing the algebraic method.
%\todo{talk about equilibrium routing, etc. and relationship to this work}

The idea that Dijkstra's algorithm can be generalised to an algorithm that solves a matrix fixpoint equation was first explored by Sobrinho~\cite{sobrinho_algebra_2001}, with Mohri later presenting a general semiring framework for shortest path algorithms~\cite{mohri:semiring:2002} where the underlying semiring and queing discipline used by the algorithm are abstracted over, with Dijkstra being a special case.
Griffin and Sobrinho explored the solutions found by the generalised algorithm whenever distributivity is not assumed~\cite{sobrinho_routing_2010}, with Griffin later producing an unpublished partial formalisation in Coq of some of these ideas, with 10 axiomatised statements~\cite{griffin:equilibrium-coq}.
Our work builds on these latter ideas, but goes beyond it in several ways: we fully specify the properties of the algebraic structure assumed, obtaining the notion of a `Sobrinho Algebra`, give a concrete implementation of the algorithm, and mechanically verify its correctness with a clean-slate axiom-free proof.

\paragraph{Acknowledgements} We thank Timothy G.~Griffin for helpful comments when preparing this paper, including the suggestion of the name `Sobrinho Algebra'.
The second author is employed on EPSRC grant EP/K008528 (`Rigorous Engineering for Mainstream Systems' Programme Grant).
