In this paper we have presented a purely functional implementation of a generalised shortest path algorithm, and proved it correct.
We have made extensive use of dependent types, and some of Agda's more advanced features, such as induction-recursion, to help structure the implementation and proof.

\paragraph{Related Work}
Remarkably, despite the algorithm being well-known, Chen seems to have been the first to verify the correctness of Dijkstra's Algorithm in any proof assistant, when he produced a Mizar implementation in 2003~\cite{chen:dijkstra:2003}.
Later, Moore and Zhang verified Dijkstra's Algorithm in ACL2~\cite{moore:proof-pearl:2005}.
Gordon, Hurd, and Slind verified Dijkstra's reachability algorithm in HOL4 as part of a wider formalisation of the Accellera Property Specification Language~\cite{gordon:executing:2003}.
Fleury verified Floyd's all-pairs shortest path algorithm in Coq~\cite{fleury:implantation:1990}, and in unpublished work, Paulin and Fill\^iatre later verified Floyd's algorithm again in imperative form with the aid of an additional tool, also in Coq.
Nordhoff and Lammich verified Dijkstra's algorithm in Isabelle/HOL as a showcase of the Isabelle refinement and collections frameworks~\cite{nordhoff-dijkstra-2012}.
Lammich also later implemented and verified an imperative version of Dijkstra's Algorithm in Isabelle~\cite{lammich:refinement:2015}.
All of these prior formalisations take a very concrete, very classical approach to shortest path algorithms and their proofs of correctness, largely following textbook treatments, in contrast to our work presented in this paper.

%\todo{talk about equilibrium routing, etc. and relationship to this work}

The idea that Dijkstra's algorithm can be generalized to an algorithm that solves a matrix fixpoint equation was first explored by Sobrinho~\cite{sobrinho_algebra_2001}. Griffin and Sobrinho explored the solutions found by the generalized algorithm when distributivity is not assumed~\cite{sobrinho_routing_2010}. Our work builds on their ideas, but goes beyond it in several key aspects: we spell out in detail the properties of the algebraic structure, give a concrete implementation of the algorithm and a mechanically verified correctness proof in Agda.

Our implementation and all supporting files are available anonymously from a public \texttt{git} repository~\cite{markert_formalised_2015}.
Documentation for type checking the formalisation is available in the repository.
The formalisation consists of approximately 2,400 lines of Agda and was developed using Agda~2.4.2.1 and~2.4.2.2 and Standard Library version~0.9.

\paragraph{Acknowledgments}
We thank Timothy G.~Griffin for helpful comments when preparing this paper.
The second author is employed on EPSRC grant EP/K008528 (REMS Programme Grant).
