\begin{figure}[t]
\centering
\begin{tabular}{c||l@{\;\;\;}|l}
\textbf{Operation} & \textbf{Semiring} & \textbf{Sobrinho Algebra} \\
\midrule
\AgdaFunction{\_+\_} & Associative & Associative \\
                 & Commutative & Commutative \\
                 & Identity: \AgdaField{0\#} & Identity: \AgdaField{0\#} \\
                 & ---                      & Selective \\
                 & ---                      & Zero: \AgdaField{1\#} \\
\midrule
\AgdaFunction{\_*\_} & Associative & --- \\
                 & Identity: \AgdaField{1\#} & Left identity: \AgdaField{1\#} \\
                 & Zero: \AgdaField{0\#}     & --- \\
\midrule
\AgdaFunction{\_*\_} and \AgdaFunction{\_+\_} & \AgdaFunction{\_*\_} distributes over \AgdaFunction{\_+\_} &
                   \AgdaFunction{\_+\_} absorbs \AgdaFunction{\_*\_} \\
\bottomrule
\end{tabular}
\label{tab.path.algebra}
\vspace{6pt}
\caption{Comparing the algebraic properties of a Semiring and a Sobrinho Algebra.}
\label{fig.path.algebra}
\end{figure}

Fix a carrier set $S$ and an equivalence relation $\_{≈}\_$.
We call a binary operation on $S$ \emph{selective} whenever $x \bullet y ≈ x$ or $x \bullet y ≈ y$ for any $x$ and $y$.
Intuitively, a selective binary operation denotes a `choice' between any two elements.
With this definition in mind, we call a 6-tuple $\langle S, \_{≈}\_, \_{+}\_, \_{*}\_, 0, 1 \rangle$ a `Sobrinho Algebra' whenever:
\begin{itemize}
\item
$\langle S, \_{≈}\_, \_{+}\_, 0 \rangle$ forms a commutative monoid,
\item
$1$ is a left identity for multiplication, and a left- and right zero for addition,
\item
Addition is selective, and addition absorbs multiplication,
\item
The usual closure and congruence properties for the unit elements and operations apply.
\end{itemize}
In Figure~\ref{fig.path.algebra}, we provide a comparison between the operations and laws of a Sobrinho Algebra and the more familiar notion of a Semiring, algebraic structures themselves often used in work on algebraic routing.

Following established convention, we capture the notion of a Sobrinho Algebra as an Agda record named \AgdaRecord{SobrinhoAlgebra}.
We call the carrier type, corresponding to the carrier set $S$ above, of a Sobrinho Algebra, \AgdaField{Carrier}, obtaining the closure properties mentioned above for `free' as a side-effect of Agda's typing discipline, and assume that there exists a decidable setoid equivalence relation on elements of this type, \AgdaField{\_≈\_}.
We use the names \AgdaField{1\#} and \AgdaField{0\#} for our two identity elements of the algebra within Agda, in order to avoid clashing with Agda's in-built numeral parsing notation for natural numbers.
